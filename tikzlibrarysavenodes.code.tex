\RequirePackage{expl3, l3keys2e, xparse}

\ExplSyntaxOn

% We save our information in a ``property list'', which is L3's
% version of an associative array or dictionary.  They keys will give
% the ability to store several groups of nodes and restore them at
% will.
\prop_new:N \g__sn_prop
% We'll need a couple of spare token lists
\tl_new:N \l__sn_tmpa_tl
\tl_new:N \l__sn_tmpb_tl

\tl_new:N \l__open_bracket_tl
\tl_set:Nn  \l__open_bracket_tl {[} %]
\tl_new:N \l__sn_group_tl
\clist_new:N \l__sn_nodes_clist

\bool_new:N \l__sn_file_bool

% Dimensions for shifting
\dim_new:N \l__sn_x_dim
\dim_new:N \l__sn_y_dim
\dim_new:N \l__sn_xa_dim
\dim_new:N \l__sn_ya_dim
\tl_new:N \l__sn_centre_tl

\tl_new:N \l__sn_transformation_tl
\tl_set:Nn \l__sn_transformation_tl {{1}{0}{0}{1}{0pt}{0pt}}

% Set up a stream for saving the nodes data to a file
\iow_new:N \g__sn_stream
\iow_open:Nn \g__sn_stream {\jobname.nodes}

\AtEndDocument
{
  \ExplSyntaxOn
  \iow_close:N \g__sn_stream
  \ExplSyntaxOff
}

% LaTeX3 wrappers around some PGF functions (to avoid @-catcode issues)
\makeatletter
\cs_set_eq:NN \tikz_set_node_name:n \tikz@pp@name
\cs_set_eq:NN \tikz_fig_must_be_named: \tikz@fig@mustbenamed

\cs_new_nopar:Npn \tikz_scan_point:n #1
{
  \tikz@scan@one@point\pgfutil@firstofone#1\relax
}

\cs_new_nopar:Npn \tikz_scan_point:NNn #1#2#3
{
  \tikz@scan@one@point\pgfutil@firstofone#3\relax
  \dim_set_eq:NN #1 \pgf@x
  \dim_set_eq:NN #2 \pgf@y
}

\makeatother

\cs_generate_variant:Nn \tikz_scan_point:n {V}
\cs_generate_variant:Nn \tikz_scan_point:NNn {NNV}


% This is the command that actually does the work.  It constructs a
% token list which contains the code that will restore the node data
% when invoked.  The two arguments are the token list to store this in
% and a comma separated list of the node names to be saved.

\cs_new_nopar:Npn \save_nodes:Nn #1#2
{
  % Clear our token lists
  \tl_clear:N \l__sn_tmpa_tl
  % Set the centre of the picture
  \tikz_scan_point:NNn \l__sn_x_dim \l__sn_y_dim {(current~ bounding~ box.center)}
  \dim_set:Nn \l__sn_x_dim {-\l__sn_x_dim}
  \dim_set:Nn \l__sn_y_dim {-\l__sn_y_dim}
  \tl_set:Nx \l__sn_centre_tl {{1}{0}{0}{1}{\dim_use:N \l__sn_x_dim}{\dim_use:N \l__sn_y_dim}}
  
  % Iterate over the list of node names
  \clist_map_inline:nn {#2}
  {
    % Test to see if the node has been defined
    \tl_if_exist:cT {pgf@sh@ns@##1}
    {
      % The node information is stored in a series of macros of the form
      % \pgf@sh@XX@nodename where XX is one of the following.
      \clist_map_inline:nn {ns,np,ma,pi}
      {
        % Our token list will look like:
        %
        % \tl_set:cn {pgf@sh@XX@nodename} {<current contents of that macro>}
        %
        % This will restore \pgf@sh@XX@nodename to its current value
        % when this list is invoked.
        %
        % This part puts the \tl_set:cn {pgf@sh@XX@nodename} in place
        \tl_put_right:Nn \l__sn_tmpa_tl
        {
          \tl_gset:cn {pgf@sh@####1@ \tikz_set_node_name:n{##1} }
        }
        % Now we put the current contents in place.  We're doing this in
        % an expansive context to get at the contents.  The \exp_not:v
        % part takes the current value of \pgf@sh@XX@nodename and puts
        % it in place, preventing further expansion.
        \tl_if_exist:cTF {pgf@sh@####1@##1}
        {
          \tl_put_right:Nx \l__sn_tmpa_tl {{\exp_not:v {pgf@sh@####1@ \tikz_set_node_name:n {##1}}}}
        }
        {
          \tl_put_right:Nx \l__sn_tmpa_tl {{}}
        }
      }
      \tl_put_right:Nn \l__sn_tmpa_tl
      {
        \tl_gset:cn {pgf@sh@nt@ \tikz_set_node_name:n{##1} }
      }
      \compose_transformations:NVv \l__sn_tmpb_tl \l__sn_centre_tl {pgf@sh@nt@##1}
      \tl_put_right:Nx \l__sn_tmpa_tl {{\exp_not:V \l__sn_tmpb_tl}}
      \tl_put_right:Nn \l__sn_tmpa_tl {\transform_node:Nn \l__sn_transformation_tl {\tikz_set_node_name:n{##1}} }
    }
  }
  % Once we've assembled our token list, we store it in the given
  % token list
  \tl_set_eq:NN #1 \l__sn_tmpa_tl
}

% Save the nodes to a list, given a key
\cs_new_nopar:Npn \save_nodes_to_list:nn #1#2
{
  \save_nodes:Nn \l__sn_tmpa_tl {#2}
  \prop_gput:NnV \g__sn_prop {#1} \l__sn_tmpa_tl
}

% Save the nodes to a file
\cs_new_nopar:Npn \save_nodes_to_file:n #1
{
  \save_nodes:Nn \l__sn_tmpa_tl {#1}
  % Save the token list to the aux file so that on reading the aux
  % file, we restore the node definitions
  \iow_now:Nx \g__sn_stream
  {
    \iow_newline:
    \exp_not:V \l__sn_tmpa_tl
    \iow_newline:
  }
}

\cs_generate_variant:Nn \save_nodes_to_list:nn {VV}

\cs_generate_variant:Nn \save_nodes_to_file:n {V}

\cs_new_nopar:Npn \restore_nodes_from_list:n #1
{
  % Restoring nodes is simple: look in the property list for the key
  % and if it exists, invoke the macro stored there.
  \prop_get:NnNT \g__sn_prop {#1} \l__sn_tmpa_tl
  {
    \tl_use:N \l__sn_tmpa_tl
  }
}

\cs_new_nopar:Npn \restore_nodes_from_file:n #1
{
  \file_if_exist:nT {#1.nodes}
  {
    \ExplSyntaxOn
    \file_input:n {#1.nodes}
    \ExplSyntaxOff
  }
}

% Compose PGF transformations #2 * #3, storing the result in #1

% I think the PGF Manual might be incorrect.  It implies that the
% matrix is stored row-major, but experimentation implies column-major.
%
% That is, {a}{b}{c}{d}{s}{t} is:
%
% [ a c ]
% [ b d ]

\cs_new_nopar:Npn \compose_transformations:Nnn #1#2#3
{
  \tl_gset:Nx #1
  {
    {\fp_eval:n {
        \tl_item:nn {#2} {1}
        * \tl_item:nn {#3} {1}
        +
        \tl_item:nn {#2} {3}
        * \tl_item:nn {#3} {2}
      }
    }
    {\fp_eval:n {
        \tl_item:nn {#2} {2}
        * \tl_item:nn {#3} {1}
        +
        \tl_item:nn {#2} {4}
        * \tl_item:nn {#3} {2}
      }
    }
    {\fp_eval:n {
        \tl_item:nn {#2} {1}
        * \tl_item:nn {#3} {3}
        +
        \tl_item:nn {#2} {3}
        * \tl_item:nn {#3} {4}
      }
    }
    {\fp_eval:n {
        \tl_item:nn {#2} {2}
        * \tl_item:nn {#3} {3}
        +
        \tl_item:nn {#2} {4}
        * \tl_item:nn {#3} {4}
      }
    }
    {\fp_to_dim:n {
        \tl_item:nn {#2} {1}
        * \tl_item:nn {#3} {5}
        +
        \tl_item:nn {#2} {3}
        * \tl_item:nn {#3} {6}
        +
        \tl_item:nn {#2} {5}
      }
    }
    {\fp_to_dim:n {
        \tl_item:nn {#2} {2}
        * \tl_item:nn {#3} {5}
        +
        \tl_item:nn {#2} {4}
        * \tl_item:nn {#3} {6}
        +
        \tl_item:nn {#2} {6}
      }
    }
  }
}

\cs_generate_variant:Nn \compose_transformations:Nnn {cVv,NVv,NVn,NvV,NnV}

\cs_new_nopar:Npn \transform_node:Nn #1#2
{
  \compose_transformations:cVv {pgf@sh@nt@#2} #1 {pgf@sh@nt@#2}
}

\cs_new_nopar:Npn \set_transform_from_node:n #1
{
  \tl_set_eq:Nc \l__sn_transformation_tl {pgf@sh@nt@#1}
  \tikz_scan_point:NNn \l__sn_x_dim \l__sn_y_dim {(#1.center)}
    
  \dim_set:Nn \l__sn_x_dim {\l__sn_x_dim - \tl_item:cn {pgf@sh@nt@#1}{5}}
  \dim_set:Nn \l__sn_y_dim {\l__sn_y_dim - \tl_item:cn {pgf@sh@nt@#1}{6}}
    
  \compose_transformations:NnV  \l__sn_transformation_tl {{1}{0}{0}{1}{\dim_use:N \l__sn_x_dim}{\dim_use:N \l__sn_y_dim}} \l__sn_transformation_tl
}

\cs_generate_variant:Nn \set_transform_from_node:n {v}

\tikzset{
  set~ node~ group/.code={
    \tl_set:Nn \l__sn_group_tl {#1}
    \pgfkeysalso{
      execute~ at~ end~ scope={
        \maybe_save_nodes:
      }
    }
  },
  % Are we saving to a file?
  save~ nodes~ to~ file/.code={
    \tl_if_eq:nnTF {#1}{false}
    {
      \bool_set_false:N \l__sn_file_bool
    }
    {
      \bool_set_true:N \l__sn_file_bool
    }
    \pgfkeysalso{
      execute~ at~ end~ scope={
        \maybe_save_nodes:
      }
    }
  },
  % Append current node to the list of nodes to be saved
  save~ node/.code={
    \tikz_fig_must_be_named:
    \pgfkeysalso{append~ after~ command={
        \pgfextra{
          \clist_gput_right:Nv \l__sn_nodes_clist {tikz@last@fig@name}
        }
      }
    }
  },
  % Restore nodes from file
  restore~ nodes~ from~ file/.code={
    \tikz_fig_must_be_named:
    \pgfkeysalso{append~ after~ command={
        \pgfextra{
          \scope
          \set_transform_from_node:v {tikz@last@fig@name}
          \split_argument:NNn \tikzset \restore_nodes_from_file:n {#1}
          \endscope
        }
      }
    }
  },
  % Restore nodes from list
  restore~ nodes~ from~ list/.code={
    \tikz_fig_must_be_named:
    \pgfkeysalso{append~ after~ command={
        \pgfextra{
          \scope
          \set_transform_from_node:v {tikz@last@fig@name}
          \split_argument:NNn \tikzset \restore_nodes_from_list:n {#1}
          \endscope
        }
      }
    }
  }
}
\cs_generate_variant:Nn \clist_gput_right:Nn {Nv}

\cs_new_nopar:Npn \split_argument:NNn #1#2#3
{
  \tl_set:Nx \l__sn_tmpa_tl {\tl_head:n {#3}}
  \tl_if_eq:NNTF \l__sn_tmpa_tl \l__open_bracket_tl
  {
    \split_argument_aux:NNp #1#2#3
  }
  {
    #2 {#3}
  }
}

\cs_new_nopar:Npn \split_argument_aux:NNp #1#2[#3]#4
{
  #1 {#3}
  #2 {#4}
}

\cs_new_nopar:Npn \maybe_save_nodes:
{
  \clist_if_empty:NF \l__sn_nodes_clist
  {
    \bool_if:NTF \l__sn_file_bool
    {
      \save_nodes_to_file:V \l__sn_nodes_clist
    }
    {
      \tl_if_empty:NF \l__sn_group_tl
      {
        \save_nodes_to_list:VV \l__sn_group_tl \l__sn_nodes_clist
      }
    }
    \clist_gclear:N \l__sn_nodes_clist
  }
}

\ExplSyntaxOff
